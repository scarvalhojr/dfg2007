\documentclass[gi]{techbrief}
\usepackage[latin1]{inputenc}
\usepackage{german}

\setlength{\textheight}{24cm}
\from{Prof. Dr. Jens Stoye}
\phone{3852}
\email{stoye}
\signature{Jens Stoye}
\place{}
\date{10.\ Januar 2007}

\begin{document}
\begin{letter}{An die\\
               Gesch�ftsstelle der DFG\\
               Bereich ING 6\\
               Kennedyallee 40\\[1ex]
               53175 Bonn}

\subject{\textbf{Antrag auf Gew�hrung einer Sachbeihilfe
                 "`Engineering von komprimierten Text-Indexstrukturen"'
                 im Rahmen des Schwerpunktprogramms "`Algorithm Engineering"'
                 (SPP 1307)}}

\opening{Sehr geehrte Frau Sonntag,}

Hiermit m�chte ich den in der Anlage befindlichen
Antrag auf Gew�hrung einer Sachbeihilfe
"`Algorithmen zu Genomstruktur und Genomdynamik"'
im Rahmen des Schwerpunktprogramms
"`Algorithm Engineering"'
bei der Deutschen Forschungsgemeinschaft einreichen.

Falls der Antrag unvollst�ndig sein sollte oder
Sie weitere Informationen ben�tigen,
k�nnen Sie sich jederzeit gerne an mich wenden

\closing{Mit freundlichen Gr��en}

\vspace{5ex}
Anlagen\\
-- Antragstext in f�nffacher Ausfertigung\\
      -- Lebenslauf des Antragsstellers (DFG-Vordruck 10.04)
         in dreifacher Ausfertigung\\
      -- Vollst�ndiges Publikationsverzeichnis des Antragstellers
         in dreifacher Ausfertigung\\
      -- Kopien von 4 Ver�ffentlichungen
         in dreifacher Ausfertigung\\
      -- CD-ROM mit elektronischen Fassungen aller Antragsdokumente:\\[.3ex]
      ~~~~ - \verb|antragstext.pdf| (Antragstext)\\
      ~~~~ - \verb|DFG_Vordruck_10_04.rtf| (DFG-Vordruck 10.04)\\
      ~~~~ - \verb|publikationsliste.pdf| (Publikationsliste)\\
      ~~~~ - \verb|nar2001.pdf| (Publikation Kurtz \textit{et al.}, 2001)\\
      ~~~~ - \verb|spe2003.pdf| (Publikation Giegerich \textit{et al.}, 2003)\\
      ~~~~ - \verb|jcb2006.pdf| (Publikation Rasmussen \textit{et al.}, 2006)\\
      ~~~~ - \verb|spe2007.pdf| (Publikation Sch�rmann und Stoye, im Druck)
\end{letter}
\end{document}
